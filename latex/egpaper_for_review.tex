\documentclass[10pt,twocolumn,letterpaper]{article}

\usepackage{cvpr}
\usepackage{times}
\usepackage{epsfig}
\usepackage{graphicx}
\usepackage{amsmath}
\usepackage{amssymb}

% Include other packages here, before hyperref.

% If you comment hyperref and then uncomment it, you should delete
% egpaper.aux before re-running latex.  (Or just hit 'q' on the first latex
% run, let it finish, and you should be clear).
\usepackage[pagebackref=true,breaklinks=true,letterpaper=true,colorlinks,bookmarks=false]{hyperref}

\cvprfinalcopy % *** Uncomment this line for the final submission

\def\cvprPaperID{****} % *** Enter the CVPR Paper ID here
\def\httilde{\mbox{\tt\raisebox{-.5ex}{\symbol{126}}}}

% Pages are numbered in submission mode, and unnumbered in camera-ready
\ifcvprfinal\pagestyle{empty}\fi
\begin{document}

%%%%%%%%% TITLE
\title{Project Report : CS 7643}

\author{First Author\\
Institution1\\
Institution1 address\\
{\tt\small firstauthor@i1.org}
% For a paper whose authors are all at the same institution,
% omit the following lines up until the closing ``}''.
% Additional authors and addresses can be added with ``\and'',
% just like the second author.
% To save space, use either the email address or home page, not both
\and
Second Author\\
Institution2\\
First line of institution2 address\\
{\tt\small secondauthor@i2.org}
}
\maketitle
%\thispagestyle{empty}

%%%%%%%%% ABSTRACT
\begin{abstract}
   In this work, we investigate the efficacy of various adapter architectures on the SuperGLUE benchmark tasks and a supplementary news classification dataset, comparing their performance to traditional fine-tuning on DistilBERT, Electra, and BART models. We implemented and evaluated several state-of-the-art adapters, including sequential bottleneck, stacked sequential bottleneck Mix and Match, iA3, LoRA, Prefix Tuning, Compacter++, Prompt Tuning, and UniPELT. Our analysis reveals performance differences across adapter architectures, highlighting their effectiveness relative to full fine-tuning. The results from the news classification task further support our findings, demonstrating adapters as efficient and flexible alternatives to fine-tuning. This study provides valuable insights and guidelines for selecting and implementing adapters in diverse NLP applications.
\end{abstract}

%%%%%%%%% BODY TEXT
\section{Introduction}
\subsection{Problem}
We aimed to compare different methods for adapting pre-trained language models like DistilBERT, Electra, and BART to new tasks without changing all their parameters. These methods, called adapters, are small, trainable components added to the models. They are supposed to be more efficient and flexible than traditional fine-tuning, which adjusts all model parameters. We tested several adapter designs on tasks from the SuperGLUE benchmark and a news classification task to see how well they work compared to fine-tuning the whole model. Our objective was to determine which adapter designs perform best and understand their advantages over traditional methods.

We aimed to compare different methods for adapting pre-trained language models like BERT, Electra, and BART to new tasks without changing all their parameters. These methods, called adapters, are small, trainable components added to the models and are more efficient and flexible than traditional fine-tuning, which adjusts all model parameters. While previous research by Houlsby et al. (2019) used BERT and Gururangan et al. (2020) used RoBERTa, no one has explored adapters on models like DistilBERT, Electra, and BART. Additionally, while Houlsby’s work focused on the GLUE benchmark, we tested on SUPERGLUE, the latest state-of-the-art benchmark. Our objective was to determine which adapter designs perform best and understand their advantages over traditional methods.

Researchers and practitioners in natural language processing (NLP) care about improving how we adapt pre-trained language models to new tasks. If our approach using adapters is successful, it will significantly reduce the computational resources and time needed for fine-tuning. This efficiency means that adapting models to new tasks will become faster, cheaper, and more accessible, particularly for researchers with limited resources. It will enable more widespread use of powerful language models in various applications, fostering innovation and advancing the field of NLP by making state-of-the-art tools available to a broader audience.

\subsection{Datasets}
We used two main datasets for our experiments: the SuperGLUE benchmark and a news classification dataset from Kaggle. SuperGLUE is a collection of diverse and challenging NLP tasks designed to evaluate the performance of language models on understanding and reasoning. It includes tasks such as question answering, natural language inference, and coreference resolution, providing a comprehensive assessment of model capabilities. By leveraging SuperGLUE, we aimed to rigorously test how well different adapter architectures can adapt pre-trained models to a variety of complex tasks.

In addition to SuperGLUE, we used a news classification dataset from Kaggle. This dataset consists of news articles labeled by category, which we used to further evaluate the adaptability and efficiency of our methods. By including a real-world dataset like this, we ensured our approach was tested in practical scenarios beyond the standard benchmarks. This combination of datasets allowed us to thoroughly assess the performance and advantages of adapters across a wide range of tasks and data types.

\subsection{Adapters}
Sequential Bottleneck is a method for model adaptation that introduces a series of bottleneck layers within the model’s architecture. These bottleneck layers, typically comprising low-dimensional projections, help in reducing the computational burden and memory usage while still capturing essential information for task-specific fine-tuning. This approach aims to efficiently adapt large pre-trained models by inserting a compact sequence of layers to handle the task-specific adjustments.

Mix and Match is an adaptation technique that combines multiple adapter modules in a flexible manner, allowing different modules to be applied or mixed for various tasks. By blending different types of adapters, this method optimizes performance across a range of tasks, leveraging the strengths of each adapter type while maintaining efficiency and minimizing the need for extensive retraining.

iA3 (Improved Adapter Architecture with Activation and Attention) enhances standard adapter methods by integrating improved activation functions and attention mechanisms. This advanced adapter design aims to boost the learning capacity and overall performance of the adapters, making them more effective in handling complex tasks while retaining a relatively small number of parameters.

LoRA (Low-Rank Adaptation) introduces a low-rank approximation into the weight updates of pre-trained models. By decomposing the weight matrices into low-rank components, LoRA reduces the number of parameters that need to be updated during fine-tuning. This approach enables efficient adaptation of large models with a significant reduction in computational and storage requirements.

Compacter++ is an extension of the Compacter method, designed to further enhance parameter efficiency and representation capacity. It leverages advanced low-rank tensor decomposition techniques and refined parameter sharing strategies to achieve more compact and efficient model adaptation, offering improved performance with minimal additional parameters.

UniPELT (Unified Prompt and Embedding Learning Techniques) combines prompt tuning with embedding learning into a unified framework. This method optimizes both the prompts used to guide model behavior and the embeddings involved in processing inputs, resulting in effective task adaptation with minimal adjustments to the underlying model. By integrating these techniques, UniPELT enhances performance across a variety of tasks while maintaining efficiency.

Prompt Tuning Adapters focus on adjusting input prompts rather than altering the model’s weights. This approach fine-tunes the prompts to better align with specific tasks, enabling task-specific adaptation with a small number of parameters. It allows for efficient customization of pre-trained models by modifying the prompts used to guide the model’s responses.
%-------------------------------------------------------------------------
%------------------------------------------------------------------------
\section{Approach}
To address our problem, we used HuggingFace's Transformers library along with the Adapters library. The Adapters repository provided code for running simple sequential bottleneck adapters on GLUE tasks, but we needed to extend this code to fit our requirements. Specifically, we added support for SuperGLUE, incorporating tasks like [insert classification tasks here]. We also extended the repository to include various adapter architectures such as sequential bottleneck, stacked sequential bottleneck Mix and Match, iA3, LoRA, Prefix Tuning, Compacter++, Prompt Tuning, and UniPELT. Additionally, we modified the code to return our desired metrics and enabled it to handle traditional fine-tuning on the same tasks for comparison.

We believed our approach would be successful because adapters offer a more efficient and flexible way to adapt pre-trained models to new tasks, requiring fewer resources than full fine-tuning. By keeping our machine learning hyperparameters constant, we ensured a fair comparison between fine-tuning and various adapter architectures. We ran each configuration for 10 epochs, maintaining consistency in learning rate, batch size, and other relevant hyperparameters to isolate the effect of the adapter architecture itself. This approach allowed us to leverage the strengths of different adapter designs and rigorously evaluate their performance against traditional fine-tuning. The advantage of this methodology is that it highlights how different adapter architectures can provide significant resource savings without compromising on performance.

After running the experiments, we plotted the training curves for each adapter on each task, as well as for fine-tuning. We compared the final accuracy and time taken by each approach. This comprehensive analysis provided clear insights into the efficiency and effectiveness of adapters, demonstrating their potential as a viable alternative to full model fine-tuning in NLP tasks. By evaluating a wide range of adapter architectures, we contributed new knowledge to the field, offering practical guidelines for selecting and implementing adapters in various NLP applications.

We anticipated that some tasks would take a long time to complete due to the large datasets and the inherent complexity of the tasks. To mitigate this, we utilized GPUs to speed up the training process and carefully planned our training schedules ahead of time to ensure efficient resource use. Additionally, we expected to face challenges with preprocessing the data for each specific task, as different tasks in the SuperGLUE benchmark have unique requirements. We adjusted our code accordingly to handle these preprocessing differences, ensuring that each task was processed correctly for both the adapters and fine-tuning methods.

One of the challenges we encountered was that some adapter architectures were not well-suited for certain tasks. For example, our initial attempt using a simple sequential bottleneck adapter on the BoolQ task ran successfully but did not surpass the performance of traditional fine-tuning. This outcome was expected, as the sequential bottleneck is a basic adapter architecture. To address this, we explored more advanced adapter architectures such as stacked sequential bottleneck Mix and Match, iA3, and LoRA, which showed improved performance. Additionally, we faced issues with integrating all adapter types into the existing codebase, which required significant modifications to ensure compatibility and correct metric reporting. Through iterative testing and code adjustments, we managed to overcome these obstacles and achieve a comprehensive evaluation of the different adapter architectures.


\textbf{Important: Mention any code repositories (with citations) or other sources that you used, and specifically what changes you made to them for your project. }

\section{Experiments and Results}

(10 points) How did you measure success? What experiments were used? What were the results, both quantitative and qualitative? Did you succeed? Did you fail? Why? Justify your reasons with arguments supported by evidence and data.

\textbf{Important: This section should be rigorous and thorough. Present detailed information about decision you made, why you made them, and any evidence/experimentation to back them up. This is especially true if you leveraged existing architectures, pre-trained models, and code (i.e. do not just show results of fine-tuning a pre-trained model without any analysis, claims/evidence, and conclusions, as that tends to not make a strong project). }

%-------------------------------------------------------------------------
\section{Other Sections}

\begin{table*}
\begin{center}
\begin{tabular}{|l|c|p{8cm}|}
\hline
Student Name & Contributed Aspects & Details \\
\hline\hline
Team Member 1 & Data Creation and Implementation & Scraped the dataset for this project and trained the CNN of the encoder. Implemented attention mechanism to improve results. \\
Muhammad Aleem & Writing base script &Implemented the base script to run base experiments on SuperGLUE tasks with the 5 adapters mentioned in the paper \\
Team Member 3 & Data Creation and Implementation & Scraped the dataset for this project and trained the CNN of the encoder. Implemented attention mechanism to improve results. \\
Team Member 4 & Implementation and Analysis & Trained the LSTM of the encoder and analyzed the results. Analyzed effect of number of nodes in hidden state.  Implemented Convolutional LSTM. \\
\hline
\end{tabular}
\end{center}
\caption{Contributions of team members.}
\label{tab:contributions}
\end{table*}



You are welcome to introduce additional sections or subsections, if required, to address the following questions in detail. 

(5 points) Appropriate use of figures / tables / visualizations. Are the ideas presented with appropriate illustration? Are the results presented clearly; are the important differences illustrated? 

(5 points) Overall clarity. Is the manuscript self-contained? Can a peer who has also taken Deep Learning understand all of the points addressed above? Is sufficient detail provided? 

(5 points) Finally, points will be distributed based on your understanding of how your project relates to Deep Learning. Here are some questions to think about: 

What was the structure of your problem? How did the structure of your model reflect the structure of your problem? 

What parts of your model had learned parameters (e.g., convolution layers) and what parts did not (e.g., post-processing classifier probabilities into decisions)? 

What representations of input and output did the neural network expect? How was the data pre/post-processed?
What was the loss function? 

Did the model overfit? How well did the approach generalize? 

What hyperparameters did the model have? How were they chosen? How did they affect performance? What optimizer was used? 

What Deep Learning framework did you use? 

What existing code or models did you start with and what did those starting points provide? 

Briefly discuss potential future work that the research community could focus on to make improvements in the direction of your project's topic.


%-------------------------------------------------------------------------

\section{Work Division}

The delegation of tasks among team members was crucial for the successful execution of the project. Muhammad Aleem initially found the base code for GLUE tasks using a single adapter and extended it to include SuperGLUE and other adapters. This foundational work set the stage for further development and integration of additional methods. [insert rest of the team work]
%-------------------------------------------------------------------------
\subsection{The ruler}
The \LaTeX\ style defines a printed ruler which should be present in the
version submitted for review.  The ruler is provided in order that
reviewers may comment on particular lines in the paper without
circumlocution.  If you are preparing a document using a non-\LaTeX\
document preparation system, please arrange for an equivalent ruler to
appear on the final output pages.  The presence or absence of the ruler
should not change the appearance of any other content on the page.  The
camera ready copy should not contain a ruler. (\LaTeX\ users may uncomment
the \verb'\cvprfinalcopy' command in the document preamble.)  Reviewers:
note that the ruler measurements do not align well with lines in the paper
--- this turns out to be very difficult to do well when the paper contains
many figures and equations, and, when done, looks ugly.  Just use fractional
references (e.g.\ this line is $095.5$), although in most cases one would
expect that the approximate location will be adequate.

\subsection{Mathematics}

Please number all of your sections and displayed equations.  It is
important for readers to be able to refer to any particular equation.  Just
because you didn't refer to it in the text doesn't mean some future reader
might not need to refer to it.  It is cumbersome to have to use
circumlocutions like ``the equation second from the top of page 3 column
1''.  (Note that the ruler will not be present in the final copy, so is not
an alternative to equation numbers).  All authors will benefit from reading
Mermin's description of how to write mathematics:
\url{http://www.pamitc.org/documents/mermin.pdf}.

Finally, you may feel you need to tell the reader that more details can be
found elsewhere, and refer them to a technical report.  For conference
submissions, the paper must stand on its own, and not {\em require} the
reviewer to go to a techreport for further details.  Thus, you may say in
the body of the paper ``further details may be found
in~\cite{Authors14b}''.  Then submit the techreport as additional material.
Again, you may not assume the reviewers will read this material.

Sometimes your paper is about a problem which you tested using a tool which
is widely known to be restricted to a single institution.  For example,
let's say it's 1969, you have solved a key problem on the Apollo lander,
and you believe that the CVPR70 audience would like to hear about your
solution.  The work is a development of your celebrated 1968 paper entitled
``Zero-g frobnication: How being the only people in the world with access to
the Apollo lander source code makes us a wow at parties'', by Zeus \etal.

You can handle this paper like any other.  Don't write ``We show how to
improve our previous work [Anonymous, 1968].  This time we tested the
algorithm on a lunar lander [name of lander removed for blind review]''.
That would be silly, and would immediately identify the authors. Instead
write the following:
\begin{quotation}
\noindent
   We describe a system for zero-g frobnication.  This
   system is new because it handles the following cases:
   A, B.  Previous systems [Zeus et al. 1968] didn't
   handle case B properly.  Ours handles it by including
   a foo term in the bar integral.

   ...

   The proposed system was integrated with the Apollo
   lunar lander, and went all the way to the moon, don't
   you know.  It displayed the following behaviours
   which show how well we solved cases A and B: ...
\end{quotation}
As you can see, the above text follows standard scientific convention,
reads better than the first version, and does not explicitly name you as
the authors.  A reviewer might think it likely that the new paper was
written by Zeus \etal, but cannot make any decision based on that guess.
He or she would have to be sure that no other authors could have been
contracted to solve problem B.
\medskip

\noindent
FAQ\medskip\\
{\bf Q:} Are acknowledgements OK?\\
{\bf A:} No.  Leave them for the final copy.\medskip\\
{\bf Q:} How do I cite my results reported in open challenges?
{\bf A:} To conform with the double blind review policy, you can report results of other challenge participants together with your results in your paper. For your results, however, you should not identify yourself and should not mention your participation in the challenge. Instead present your results referring to the method proposed in your paper and draw conclusions based on the experimental comparison to other results.\medskip\\

\begin{figure}[t]
\begin{center}
\fbox{\rule{0pt}{2in} \rule{0.9\linewidth}{0pt}}
   %\includegraphics[width=0.8\linewidth]{egfigure.eps}
\end{center}
   \caption{Example of caption.  It is set in Roman so that mathematics
   (always set in Roman: $B \sin A = A \sin B$) may be included without an
   ugly clash.}
\label{fig:long}
\label{fig:onecol}
\end{figure}

\subsection{Miscellaneous}

\noindent
Compare the following:\\
\begin{tabular}{ll}
 \verb'$conf_a$' &  $conf_a$ \\
 \verb'$\mathit{conf}_a$' & $\mathit{conf}_a$
\end{tabular}\\
See The \TeX book, p165.

The space after \eg, meaning ``for example'', should not be a
sentence-ending space. So \eg is correct, {\em e.g.} is not.  The provided
\verb'\eg' macro takes care of this.

When citing a multi-author paper, you may save space by using ``et alia'',
shortened to ``\etal'' (not ``{\em et.\ al.}'' as ``{\em et}'' is a complete word.)
However, use it only when there are three or more authors.  Thus, the
following is correct: ``
   Frobnication has been trendy lately.
   It was introduced by Alpher~\cite{Alpher02}, and subsequently developed by
   Alpher and Fotheringham-Smythe~\cite{Alpher03}, and Alpher \etal~\cite{Alpher04}.''

This is incorrect: ``... subsequently developed by Alpher \etal~\cite{Alpher03} ...''
because reference~\cite{Alpher03} has just two authors.  If you use the
\verb'\etal' macro provided, then you need not worry about double periods
when used at the end of a sentence as in Alpher \etal.

For this citation style, keep multiple citations in numerical (not
chronological) order, so prefer \cite{Alpher03,Alpher02,Authors14} to
\cite{Alpher02,Alpher03,Authors14}.


\begin{figure*}
\begin{center}
\fbox{\rule{0pt}{2in} \rule{.9\linewidth}{0pt}}
\end{center}
   \caption{Example of a short caption, which should be centered.}
\label{fig:short}
\end{figure*}

%------------------------------------------------------------------------
\subsection{Formatting your paper}

All text must be in a two-column format. The total allowable width of the
text area is $6\frac78$ inches (17.5 cm) wide by $8\frac78$ inches (22.54
cm) high. Columns are to be $3\frac14$ inches (8.25 cm) wide, with a
$\frac{5}{16}$ inch (0.8 cm) space between them. The main title (on the
first page) should begin 1.0 inch (2.54 cm) from the top edge of the
page. The second and following pages should begin 1.0 inch (2.54 cm) from
the top edge. On all pages, the bottom margin should be 1-1/8 inches (2.86
cm) from the bottom edge of the page for $8.5 \times 11$-inch paper; for A4
paper, approximately 1-5/8 inches (4.13 cm) from the bottom edge of the
page.

%-------------------------------------------------------------------------
\subsection{Margins and page numbering}

All printed material, including text, illustrations, and charts, must be kept
within a print area 6-7/8 inches (17.5 cm) wide by 8-7/8 inches (22.54 cm)
high.



%-------------------------------------------------------------------------
\subsection{Type-style and fonts}

Wherever Times is specified, Times Roman may also be used. If neither is
available on your word processor, please use the font closest in
appearance to Times to which you have access.

MAIN TITLE. Center the title 1-3/8 inches (3.49 cm) from the top edge of
the first page. The title should be in Times 14-point, boldface type.
Capitalize the first letter of nouns, pronouns, verbs, adjectives, and
adverbs; do not capitalize articles, coordinate conjunctions, or
prepositions (unless the title begins with such a word). Leave two blank
lines after the title.

AUTHOR NAME(s) and AFFILIATION(s) are to be centered beneath the title
and printed in Times 12-point, non-boldface type. This information is to
be followed by two blank lines.

The ABSTRACT and MAIN TEXT are to be in a two-column format.

MAIN TEXT. Type main text in 10-point Times, single-spaced. Do NOT use
double-spacing. All paragraphs should be indented 1 pica (approx. 1/6
inch or 0.422 cm). Make sure your text is fully justified---that is,
flush left and flush right. Please do not place any additional blank
lines between paragraphs.

Figure and table captions should be 9-point Roman type as in
Figures~\ref{fig:onecol} and~\ref{fig:short}.  Short captions should be centred.

\noindent Callouts should be 9-point Helvetica, non-boldface type.
Initially capitalize only the first word of section titles and first-,
second-, and third-order headings.

FIRST-ORDER HEADINGS. (For example, {\large \bf 1. Introduction})
should be Times 12-point boldface, initially capitalized, flush left,
with one blank line before, and one blank line after.

SECOND-ORDER HEADINGS. (For example, { \bf 1.1. Database elements})
should be Times 11-point boldface, initially capitalized, flush left,
with one blank line before, and one after. If you require a third-order
heading (we discourage it), use 10-point Times, boldface, initially
capitalized, flush left, preceded by one blank line, followed by a period
and your text on the same line.

%-------------------------------------------------------------------------
\subsection{Footnotes}

Please use footnotes\footnote {This is what a footnote looks like.  It
often distracts the reader from the main flow of the argument.} sparingly.
Indeed, try to avoid footnotes altogether and include necessary peripheral
observations in
the text (within parentheses, if you prefer, as in this sentence).  If you
wish to use a footnote, place it at the bottom of the column on the page on
which it is referenced. Use Times 8-point type, single-spaced.


%-------------------------------------------------------------------------
\subsection{References}

List and number all bibliographical references in 9-point Times,
single-spaced, at the end of your paper. When referenced in the text,
enclose the citation number in square brackets, for
example~\cite{Authors14}.  Where appropriate, include the name(s) of
editors of referenced books.

\begin{table}
\begin{center}
\begin{tabular}{|l|c|}
\hline
Method & Frobnability \\
\hline\hline
Theirs & Frumpy \\
Yours & Frobbly \\
Ours & Makes one's heart Frob\\
\hline
\end{tabular}
\end{center}
\caption{Results.   Ours is better.}
\end{table}

%-------------------------------------------------------------------------
\subsection{Illustrations, graphs, and photographs}

All graphics should be centered.  Please ensure that any point you wish to
make is resolvable in a printed copy of the paper.  Resize fonts in figures
to match the font in the body text, and choose line widths which render
effectively in print.  Many readers (and reviewers), even of an electronic
copy, will choose to print your paper in order to read it.  You cannot
insist that they do otherwise, and therefore must not assume that they can
zoom in to see tiny details on a graphic.

When placing figures in \LaTeX, it's almost always best to use
\verb+\includegraphics+, and to specify the  figure width as a multiple of
the line width as in the example below
{\small\begin{verbatim}
   \usepackage[dvips]{graphicx} ...
   \includegraphics[width=0.8\linewidth]
                   {myfile.eps}
\end{verbatim}
}


%-------------------------------------------------------------------------
\subsection{Color}

Please refer to the author guidelines on the CVPR 2020 web page for a discussion
of the use of color in your document.

%------------------------------------------------------------------------

%-------------------------------------------------------------------------


{\small
\bibliographystyle{ieee_fullname}
\bibliography{egbib}
}

\end{document}
